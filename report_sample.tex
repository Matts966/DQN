\documentclass[11pt,a4paper]{jsarticle}
\usepackage{amsmath,amssymb}
\usepackage{newtxtext,newtxmath}
\usepackage[dvipdfmx]{graphicx}
\usepackage[dvipdfmx]{color}
\usepackage{listings}
\lstset{%
 language={python},
 % backgroundcolor={\color[gray]{.95}},%
 tabsize=2, % tab space width
 showstringspaces=false, % don't mark spaces in strings
 basicstyle={\ttfamily},%
 %identifierstyle={\small},%
 commentstyle={\itshape},%
 keywordstyle={\bfseries},%
 %ndkeywordstyle={\small},%
 stringstyle={\ttfamily},
 %frame={tb},
 breaklines=true,
 columns=[l]{fullflexible},%
 % numbers=left,%
 % numberstyle={\small},%
 xrightmargin=0zw,%
 %xleftmargin=3zw,%
 stepnumber=1,
 numbersep=1zw,%
 lineskip=-0.5ex%
}


\title{report}
\author{氏 名: Matts966}
\date{\today}
%
\begin{document}
\maketitle
%
\section{
  作成したプログラムとその目的について
}
今回作成したプログラムは、三目並べを解くことを目的として、DoubleDQN(Deep Q Network)と
呼ばれる学習手法を、ChainerRLというライブラリを用いて実装し、Minimax法と簡単なコツを利用
したアルゴリズムと戦わせる、というものである。
添付したNumpy形式のログにもあるように、30万回強の学習回数でも、学習は収束せず、簡単なアルゴ
リズムにもまけることがある、ということがわかった。
ただ、そもそも置けない場所を選択肢に含め、それを選んだ場合にMissとして処理するという計測方法
は、条件が厳しかったかもしれない。
機械学習も、目的や細かい実装によっては、思い通りの結果を出すことができない、ということの例には
なるかもしれない。
最後には、実際に自分の手で遊んで、学習の度合を確かめられるようになっている。
以下はそのプログラムである。

\subsection{プログラム}
\label{sec:prog-list1}


\lstinputlisting[numbers=left,numberstyle=\ttfamily,xleftmargin=2zw]{dqn.py}

\end{document}
